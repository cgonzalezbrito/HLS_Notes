\documentclass{article}

\usepackage{graphicx}
\usepackage{listings}
\usepackage{color}
\usepackage[dvipsnames]{xcolor}

\renewcommand\lstlistingname{Quelltext} % Change language of section name

\definecolor{ltgray}{rgb}{0.97,0.97,0.97}


\title{My HLS Notes}
\date{2023-01-26}
\author{cgonzalezbrito}

\begin{document}
  \pagenumbering{gobble}
  \maketitle
  \newpage
  \pagenumbering{arabic}
  \section{Introduction}
  
  \iffalse TODO: \fi

  \newpage
  \section{Definitions}

  \textbf{HLS:} High-Level Synthesis\footnote{\label{HLS}https://en.wikipedia.org/wiki/High-level_synthesis}. Automated design process that takes an abstract behavioral specification of a digital system and finds a register-transfer level structure that realizes the given behavior.\\
  \textbf{RTL:} Registrer Transfer Level \footnote{\label{RTL}https://en.wikipedia.org/wiki/Register-transfer_level}. Design abstraction which models a synchronous digital circuit in terms of the flow of digital signals (data) between hardware registers, and the logical operations performed on those signals.\\
  \textbf{PL:} Programable Logic.\\
  \textbf{Throughput:} Number of specific actions executed or results obtained per unit of time.\\
  \textbf{Performance:} Higher throughput with lower power consumption.\\

  \section{Software Utilities}

  \textbf{Vitis HLS:} Vivado Design Suite + Vitis Core Development Kit\\
  \textbf{Vivado Design Suite:} For sythesis, place and route.\\
  \textbf{Vitis Core Development Kit:} For heteregeneous system-level design and application acceleration.\\



\end{document}
